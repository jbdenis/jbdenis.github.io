%
%% created with 'latex1html4file' on 2019_10_14@08:51
%
\documentclass[12pt,french]{article}
\usepackage[T1]{fontenc}
\usepackage{float}
\usepackage[utf8]{inputenc}
\usepackage[a4paper]{geometry}
\geometry{verbose,tmargin=15mm,bmargin=15mm,lmargin=15mm,rmargin=15mm}
\usepackage{graphicx}
\usepackage{babel}
\usepackage{pdflscape}
\addto\captionsfrench{
\renewcommand{\figurename}{Ph.}
}
%
\begin{document}
\title{Pour COMMENCER}
\author{}
\date{Le lundi 14 octobre 2019 à 8 heures et 51 minutes}
\maketitle
\tableofcontents
%
\section{Première Partie}
%

Tout bon texte doit comprendre une introduction


<<B>>


\begin{itemize}
\item  un
\item premier
\item deux
\item second
\item trois
\item dernier
\end{itemize}


<<N>>


\begin{enumerate}
\item  un
\item premier
\item deux
\item second
\item trois
\item dernier
\end{enumerate}


<<D>>


\begin{description}
\item [ un] premier
\item [deux] second
\item [trois] dernier
\end{description}


<<b>>


 un - premier - deux - second - trois - dernier


<<n>>


(1):  un - (2): premier - (3): deux - (4): second - (5): trois - (6): dernier


<<d>>


\textbf{ un}: premier - \textbf{deux}: second - \textbf{trois}: dernier

%
\subsection{introduction}
%

Une bonne introduction annonce la suite.

Mais elle ne doit pas dévoiler le contenu du texte.

Celui-si est réservé au développement !

%
\subsection{développement}
%
%
\subsection{conclusion}
%
%
\section{Considérations Générales}
%
%
\end{document}
%
